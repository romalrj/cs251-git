\documentclass[10.5pt, a4paper]{article}
\usepackage{graphicx}
\usepackage{amsfonts}
\usepackage[top=0.3in,left=0.2in,right=0.2in]{geometry}
\usepackage{times}
\usepackage{tabularx}
\title{3 Idiots}
\Author{Romal, Pooja and Saketh}
\begin{document}
\maketitle
\textbf{Title: 3 Idiots}
\textbf{Director: Rajkumar Hirani}
\textbf{Screenplay: Rajkumar Hirani, Vidhu Vinod Chopra, Abhijat Joshi}

Two  are searching for their long lost companion.

The film begins with the entry of our threesome in the city's elite engineering college.
 It takes the first tryst with the mandatory ragging sessions which enunciate who the leader
 of the gang is going to be: new entrant Baba Rancchoddas, as his friends fondly call him. Rancho not only leads
 his friends through the maze of India's competitive, high-pressure, rote-heavy, illogical and almost cruel
 education system, he tutors them on several life mantras too. Like, running after excellence, not success;
 questioning not blindly accepting givens; inventing and experimenting in lieu of copying and cramming;
 and essentially following your heart's calling if you truly want to make
 a difference.

Two friends are searching for their long lost companion.They revisit their college days 
and recall the memories of their friend who inspired them to think differently, 
even as the rest of the world called them "idiots". 
The high point of the film is the fact that director Rajkumar Hirani says so much, and more, without losing his sense of humour and the sheer lightness of being. The film is a laugh riot, despite being high on fundas. Certain sequences almost have you rolling in the aisle, like the ragging sequence, Omi's chamatkar/balatkar speech, the threesome's wedding crasher sequence, their mournful meal with Raju's mournful mum and Rancho's sundry demos to prove how Kareena has chosen the wrong guy for herself.

Two friends are searching for their long lost companion.They revisit their college days 
and recall the memories of their friend who inspired them to think differently, 
even as the rest of the world called them "idiots". 

Rajkumar Hirani and Abhijat Joshi script a warm and humanist indictment of India's rude-crude education system that prepares rats for a rat race rather than thinkers for a new world.
Witty and wild, the film walks away with the best comic scene of the year citation with its uproarious `balatkar' speech.Shantanu Moitra may not have forced you to pick up the music album of the film but the songs do come alive on screen, specially Zoobie-Doobie and Aal Izz Well.Chetan Bhagat's Five Point Someone literally comes alive on screen, although the film does not kowtow the book verbatim.
\textbf{Reviews by some famous critics}\\
\textbf{Subhash K. Jha }
"It's not that 3 Idiots is a flawless work of art. But it is a vital, inspiring and life-revising work of contemporary art with some heart imbued into every part. In a country where students are driven to suicide by their impossible curriculum, 3 Idiots provides hope. Maybe cinema can't save lives. But cinema, sure as hell, can make you feel life is worth living. 3 Idiots does just that, and much more. The director takes the definition of entertainment into directions of social comment without assuming that he knows best."
\begin{figure}
\includegraphic{example.jpg}
\end{figure}
\end{document}
